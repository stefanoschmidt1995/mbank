\documentclass[twocolumn,showpacs,preprintnumbers,nofootinbib,prd,
superscriptaddress,10pt]{revtex4-1}

\usepackage{amsmath,amssymb}
\usepackage{amsfonts}
\usepackage[normalem]{ulem}
\usepackage{textcomp}
\usepackage{hyperref}
\usepackage{enumitem}
\usepackage{bm}
\usepackage{afterpage}
\usepackage{graphicx}
\usepackage{psfrag}
\usepackage{mathtools}
\usepackage{tensor}
\usepackage{layouts}
\usepackage{DejaVuSans}
\usepackage{epstopdf}
\usepackage[usenames,dvipsnames]{xcolor}
\usepackage[utf8]{inputenc}
\usepackage{multirow}
\usepackage{rotating}
\usepackage{tabularx}
\usepackage{ragged2e}
\usepackage{blindtext}
\usepackage{graphicx}
\usepackage{siunitx}
	\sisetup{output-decimal-marker={.}}
	
	%some math symbols
\newcommand{\R}{\mathbb{R}}
\newcommand{\N}{\mathbb{N}}
\DeclareMathOperator{\sign}{sign}
\renewcommand{\d}[1]{\ensuremath{\operatorname{d}\!{#1}}}
%argmin and argmax
\DeclareMathOperator*{\argmax}{arg\,max}
\DeclareMathOperator*{\argmin}{arg\,min}

% comments command
\newcommand{\an}[1]{{\textcolor{orange}{\texttt{AN: #1}} }}
\newcommand{\pr}[1]{{\textcolor{cyan}{\texttt{PR: #1}} }}
\newcommand{\fm}[1]{{\textcolor{blue}{\texttt{FM: #1}} }}
\newcommand{\sschmidt}[1]{{\textcolor{red}{\texttt{SS: #1}} }}
\newcommand{\BS}[1]{{\textcolor{green}{\texttt{SB: #1}} }}
\newcommand{\oldnewtxt}[2]{\sout{#1}\textcolor{red}{#2}}

\begin{document}

	%%%%%%%%%%%%%%%%%%%%%%%%%%%%%%%%% ABSTRACT
\begin{abstract}
	See paper plan: \href{https://docs.google.com/document/d/1O8z0aDlXtV0LyrtK60vaQDzk9iiCsrjj1ThRGX1tX-0/edit}{google-doc}

\end{abstract}
	
	%%%%%%%%%%%%%%%%%%%%%%%%%%%%%%%%% TITLE
	\title{Metric bank placement}
	\author{Stefano \surname{Schmidt}}
		\email{s.schmidt@uu.nl}
        \affiliation{Nikhef, Science Park 105, 1098 XG, Amsterdam, The Netherlands}
        \affiliation{Institute for Gravitational and Subatomic Physics (GRASP),
Utrecht University, Princetonplein 1, 3584 CC Utrecht, The Netherlands}
        
        %
	\author{Sarah \surname{Caudill}}
		\email{s.e.caudill@uu.nl}
        \affiliation{Nikhef, Science Park 105, 1098 XG, Amsterdam, The Netherlands}
        \affiliation{Institute for Gravitational and Subatomic Physics (GRASP),
Utrecht University, Princetonplein 1, 3584 CC Utrecht, The Netherlands}
	\maketitle

	\tableofcontents

	%%%%%%%%%%%%%%%%%%%%%%%%%%%%%%%%% BODY 
\section{Introduction}
Usual intro \cite{PhysRevD.95.042001}. Essential features of \texttt{mbank}:
	\begin{itemize}
		\item Fast
		\item Decently accurate
		\item Useful for exploration of exotic banks
	\end{itemize}

	%%%%%%%%%%%%%%%%%%%%%%%%%%%%%%%%%
\section{Methods}

\subsection{The metric}

Description of the metric (also in appendix \ref{app:metric})

\subsection{The tiling}

Description of the tiling and how it is generated.

\subsection{Bank generation}

Description of the workflow for the bank generation.
The geometric placing method is described in appendix \ref{app:placing}

	%%%%%%%%%%%%%%%%%%%%%%%%%%%%%%%%%
\section{Validation}

\subsection{Metric accuracy}

Histogram with the actual match for points with a constant metric match

\subsection{Tiling accuracy}

Compute the match computed with the tiling vs the actual match.

\subsection{Comparison with \texttt{sbank} }

Generating 2/3 small non-precessing banks with both \texttt{mbank} and \texttt{sbank}.
Comparison based on:
	\begin{itemize}
		\item size
		\item effectualness
		\item speed
	\end{itemize}

	%%%%%%%%%%%%%%%%%%%%%%%%%%%%%%%%%
\section{Bank generation: two case studies}

Useful to present the features of these two banks generated?

\subsection{An precessing bank}
This could be the bank we use for precessing searches.

\subsection{An eccentric bank}
This could be a nonspinning, eccentric bank. It should have 500K templates for a reasonable range of masses.

\section{Final remarks and future prospects}
Usual stuff

	%%%%%%%%%%%%%%%%%%%%%%%%%%%%%%%%% ACKNOWLEDGMENTS
        \begin{acknowledgments}
         
          WRITEME...
        \end{acknowledgments}

	%%%%%%%%%%%%%%%%%%%%%%%%%%%%%%%%% APPENDIX
\appendix
\section{Details of the metric computation}\label{app:metric}
\section{Details of the geometric template placing}\label{app:placing}
	
	%%%%%%%%%%%%%%%%%%%%%%%%%%%%%%%%% BIBLIOGRAPHY
	\bibliography{biblio.bib}
	\bibliographystyle{ieeetr}

\end{document}



